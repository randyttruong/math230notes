\documentclass{report}
\usepackage{amsmath}
\usepackage{amsthm}
\usepackage{amssymb}

%%%%%%%%%%%%%%%%%%%%%%%%
\newcommand{\R}{\mathbb{R}}
%%%%%%%%%%%%%%%%%%%%%%%%
\title{Math 230-1 Notes}
\author{Randy Truong}
\begin{document}
\maketitle
\begin{sloppypar}
\tableofcontents
\chapter{12.1: Three-Dimensional Coordinate Systems}
\section{Reminders}
\begin{itemize}
  \item There are \textbf{two} MyLab Math
        assignments that are due on
        \textbf{Sunday, April 2nd, 2023}
        \begin{itemize}
          \item Three-Dimensional
                Coordinate Systems
          \item Vectors

        \end{itemize}


  \item The \textbf{first
        written homework} is
        going to be due on \textbf{Friday,
        April 12, 2023}.

  \item Remember to re-write notes in LaTeX
        for every class!
\end{itemize}

\section{Objectives}
\begin{itemize}
  \item Be able to understand and visualize
        the three-dimensional coordinate
        plane.
  \item Be able to draw basic
        objects in the three-dimensional
        coordinate plane
  \item Become fluent in the various
        attributes of the three-dimensional
        coordinate plane
  \item Be able to define a graph
        in set-builder notation
\end{itemize}

\section{Motivation}
In former calculus classes (and former math
classes in general), we have learned how to
graph different objects in two-dimensional
space. We have also learned about a particular
way of graphing objects (or interpreting
coordinates), and that has been through
coordinate grids.

\par However, in multivariable calculus, we
want to move out of these two-dimesional
coordinate systems. Instead we want to
be able to understand graphs in the
third-dimensional coordinate system in order
to tackle more complex problems.

\section{What we already know}
\subsection{Cartesian Coordinate Systems}
The Cartesian Coordinate system, which
is also known as the \textbf{rectangular
  coordinate system} is a
coordinate system in which we locate our
points based on their position in relation
to the origin of the graph, based on the
x and y axes.
\begin{itemize}

  \item For example, given the following point:

        \[ (4,5)\]

        what exactly comes to mind?
        \begin{itemize}
          \item We shift the point 4 positive
                units along the x-axis from
                the origin.
          \item We shfit the point 5
                positive units along the y-axis
                from the origin.

        \end{itemize}
\end{itemize}

\subsection{Two-dimensional coordinate systems $ \R^{2} $}
Recall that real numbers are numbers that
can be used to express one-dimensional
quantities.
\begin{itemize}
  \item This basically includes
        every single number that can be
        plotted on the number line.
\end{itemize}
We denote real numbers in mathematics with
the following symbol:
\begin{center}
  $ \R $
\end{center}
$ \R $ represents all \textbf{real,  one-dimensional
  quantities.} We can, of course,
think of this as all of the numbers and points
that exist on the number line, since
the number line only contains one-dimension,
the scalar $x$.
By comparision, whenever we see
the following notation:
\[ \R^{2}\]
This means that are observing all \textbf{real, two-dimensional quantities}.
When we say, ``real, two-dimensional
quantities,'' we are referring to all of
the points that exist in the xy-plane, or
the two-dimensional Cartesian coordinate
system.
\par By this logic, then, we know that if
$ \R^{2} $ represents a pair of real numbers,
we know that
\[ \R^{3} \]
represents a triple of all real numbers, in
the form of $ (x, y, z)$.

\subsection{Terminology in $\R^{2}$}
\theoremstyle{definition}
\newtheorem{definition}{Definition}
\begin{definition}
  Axes
\end{definition}
\textbf{Axes} represent the way that a point can
``move'' in a coordinate plane.
\begin{itemize}
  \item In the one-dimensional coordinate
        system, we can only move along the
        x-axis: $ x $.
  \item In the two-dimensional
        coordinate system, we can move
        along both the x-axis and the y-axis:
        $ (x, y) $.
\end{itemize}

\begin{definition}
  Quadrants
\end{definition}
\textbf{Quadrants} define the different
possible areas of the coordinate system a
point can exist on, which are based on
the signs of both the x and y values.
\begin{itemize}
  \item For example, there are
        four quadrants in the xy-plane, including
        \begin{itemize}
          \item Quadrant I: $ (+, +)$
          \item Quadrant II: $ (-, +)$
          \item Quadrant III: $ (-, -)$
          \item Quadrant IV: $ (+, -)$

        \end{itemize}

\end{itemize}

\section{The Three-Dimensional Coordinate
  Plane $\R^{3}$}

By what we know about the one-dimensional
coordinate system $ \R$ as $x$ and the
two-dimensional coordinate system $\R^{2}$
as $ (x, y)$, we must think of the
three-dimensional coordinate system $\R^{3}$
as $ (x, y, z)$.
\subsection{Terminology in $ \R^{3} $}

\begin{definition}
  Axes
\end{definition}

This is literally a copy of what we have
in $ \R $ and $ \R^{2} $, insofar that
we have the number of dimensions corresponding
to the exponent of $ \R $. Obviously, in
this case, since we are working in $ \R^{3} $,
we now have three dimensions to work with,
the x-axis, the y-axis, and the z-axis.

\begin{definition}
  Octants
\end{definition}
Similarly to what we had in the two-dimensional
coordinate plane, we can distinguish what
general area a point in three dimensions is
going to occupy.
\begin{itemize}
  \item There is no good way to define
        which octant is which, but we can
        visualize it as the xy-plane quadrants,
        but just duplicated for all positive
        values of z and all negative values of
        z.
\end{itemize}
\begin{definition}
  Planes
\end{definition}
\textbf{Planes} are objects that occupy
all real-numbers in two dimensions.
\par There are three planes in $ \R^{3} $
\begin{itemize}
  \item xy-plane
        \begin{itemize}
          \item We can think of this as
                all points in which $ z = 0 $.
          \item All  points
                that satisfy $ (x, y, 0) $,
                where $ x $ and $ y $ are
                real numbers.
        \end{itemize}

  \item yz-plane
        \begin{itemize}
          \item All points in which $ x = 0 $
          \item Any coordinates that satisfy
                $ (0, y , z )$ where
                $ y $ and $ z $ are
                real numbers.

        \end{itemize}
  \item xz-plane
        \begin{itemize}
          \item All points in which $ y = 0 $
          \item Any coordinates that
                satisfy $ (x, 0, z)$, where
                $ x $ and $ z $ are real
                numbers.

        \end{itemize}

\end{itemize}

\section{Set-Builder Notation}
\subsection{What are sets?}
\textbf{Sets} are just collections of different
objects in mathematics.
\begin{itemize}
  \item We can think of sets as
        containing integers, variables,
        etc\dots
\end{itemize}
They are generally notated using \textbf{curly
  braces}.
\subsection{Examples of Sets}
\[ \{ 1, 2, 3, \dots \}\]
\[ \{ a, b, c, \dots \}\]

But, how do we define what kinds of objects we are
putting into our set?
\subsection{Set Builder Notation}
\textbf{Set Builder Notation} is a type of
mathematical notation that allows us to describe what
kinds of objects are in our sets and the properties of
such objects.

\par The generally follow the following format

\[ \{ \textrm{
    \textit{variable(s)}}: \textrm{\textit{condition(s) that define the variable(s)} }\}\]
\subsection{Common Symbols in Set Builder Notation}
\newtheorem*{definition*}{Meaning}
\begin{center}
  \fbox {
    \parbox{\textwidth}{

\newtheorem{plain}{Symbol}
\begin{plain}

  \[ \R, ~ \R^{2}, ~ \R^{3}, \dots\, \R^{N}\]
\end{plain}
\begin{definition*}
  the set of real numbers in N dimensions
\end{definition*}

}}
\end{center}

\begin{center}
  \fbox{
    \parbox{\textwidth}{
      \begin{plain}
              \[ \in \]

      \end{plain}
      \begin{definition*}
                ``is an element of '' or ``in'' or ``belongs to ''
      \end{definition*}

    }
  }
\end{center}

\begin{center}
  \fbox{
    \parbox{\textwidth}{

      \begin{plain}
        \[ : ~\&~  | \]

      \end{plain}
      \begin{definition*}
          ``such that''
      \end{definition*}
    }
  }

\end{center}

\subsection{Non-mathematical examples of Set Builder Notation}

\newtheorem*{remark*}{Example}
\begin{remark*}
  \[ \{ x : \textrm{$ x $ is a left-handed guitar player} \}\]
\end{remark*}
\begin{definition*}
  The set of $ x $ such that $ x $ is a left-handed guitar
  player.
\end{definition*}
\[ \]
\begin{remark*}
  \[ \{ y ~|~ \textrm{$ y $'s name is Randy Truong} \}\]
\end{remark*}

\begin{definition*}
  The set of $ y $ such that $ y $'s name is Randy Truong.
\end{definition*}

\subsection{Mathematical Examples of Set Builder Notation}
\begin{remark*}
  \[ \{ (x, y, z) \in \R^{3} : y=0, z=0 \}\]
\end{remark*}
\begin{definition*}
  The set of all ordered triples $ (x, y, z)$ such that
  $ y $ is equal to $ 0 $ and $ z $ is equal to $ 0 $.
\end{definition*}

\begin{remark*}
  \[ \{(t, 0, 0): t \in \R \} \]
\end{remark*}
\begin{definition*}
  The set of all ordered triples $ (t, 0, 0 ) $ such that
  $ t $ is an element of real numbers (or is a real number).
\end{definition*}

\section{Drawing basic objects (points, lines, planes) in $
  R^{3}$}
Whenever we want to draw things in three dimensions, there
are a few things that we need to consider first.

\subsection{Drawing the Coordinate System and Right-Hand Rule}
Whenever we draw the three-dimensional coordinate system,
we must remember that there is a particular way in
which we draw the system. The best way to visualize
this is to use the \textbf{right-hand rule}
\begin{itemize}
  \item Our arm represents the y-axis, while our
        fingers represent the x-axis.
  \item Our thumb is always going to be pointing towards
        the z-axis.
  \item Make sure that whenever we are drawing
        a coordinate system that we are just rotating
        the system, rather than just ``mirroring'' it.
\end{itemize}
Otherwise, whenever we actually plot our points
and actually draw things in three dimensions, we need
to follow this algorithm:
\begin{enumerate}
  \item Think of what happens to the object at the origin or
        think of the shape in two dimensions.
  \item Shift the object accordingly based on the
        third dimension.
\end{enumerate}


\section{Distance Between Two Points in Three-Dimensional
  Spacde}
\subsection{Distance in $ \R^{2} $}
\newtheorem{lem}{Formula}
\begin{center}
  \fbox{
    \parbox{\textwidth} {
      \begin{lem}
        Distance in $ \R^{2} $
        \[ \textrm{let $ d $ be distance}\]
        \[ d = \sqrt{(x_{2}-x_{1})^{2}+(y_{2}-y_{1})^{2}} \]
    \end{lem}
    }
  }
\end{center}

\subsection{Distance in $ \R^{3} $}
\begin{center}
  \fbox{
    \parbox{\textwidth}{
      \begin{lem}
        Distance in $ \R^{3} $
        \[ \textrm{let $ d $ be distance}\]
        \[ d = \sqrt{(x_{2}-x_{1})^{2}+(y_{2}+y_{1})^{2}
        + (z_{2}-z_{1})^{2}}\]
      \end{lem}
    }
  }
\end{center}


\chapter{12.2: Vectors}
\section{Reminders}
TODO
\section{Objectives}
\begin{enumerate}
  \item Explain the difference between a vector
        and a point
  \item Express the vector in \textbf{component form}
        and compute its magnitude
  \item Perform elementary vector algebra using
        vector addition and scalar multiplication
  \item Produce a unit vector with a specificed
        direction
  \item Compute the midpoint of a line segment


\end{enumerate}
\section{Motivation}
TODO
\section{Scalar Who?}
TODO
\section{Vector Who?}
TODO
\subsection{Vector Notation}
TODO
\subsection{Vector Magnitude}
TODO
\subsection{Equivalency between Vectors}
TODO
\section{Position Vectors}
TODO
\section{Component Form of Vectors}
TODO
\section{Basic Vector Operations}
TODO
\subsection{Vector Addition and Subtraction}
TODO
\subsection{Scalar Multiplication}
TODO



\chapter{12.3: The Dot Product}
\section{Reminders}
TODO
\section{Objectives}
\begin{enumerate}
  \item Compute the dot product of two vectors
  \item COmpute the angle between two
        vectors in terms of the dot product
  \item Algebraically determine when two vectors
        are \textbf{orthogonal} and be able to
        geometrically define what \textbf{orthongonality}
        refers to
  \item Perform elementary vector algebra using
        properties of vector addition, scalar
        multiplication, and the dot product
  \item Algebraically compute (and geometrically
        explain/describe) the projection of a given
        vector onto another, non-zero vector
  \item Solve elemntary problems involving effective
        force and work using vector projections
\end{enumerate}

\section{Motivation}
TODO
\chapter{12.4: The Cross Product}
\section{Reminders}
TODO
\section{Objectives}
\begin{enumerate}
  \item Compute the cross product of two given
        vectors using \textbf{determinants}
  \item Geometrically interpret the
        magnitude and direction of the cross product
        of two given vectors
  \item Perform elementary vector operations
        \begin{itemize}
          \item Vector Addition
          \item Scalar Multiplication
          \item Dot Product
                \item Cross Product
        \end{itemize}

\end{enumerate}

\section{Recall}
Recall that in the last lecture, we discussed the
\textbf{dot product}, which had both an \textbf{algebraic}
definition as well as a \textbf{geometric} definition
\begin{itemize}
  \item Whenever we were finding the algebraic
        dot product, we were just multiplying the components
        and finding their sum
        \[ \vec{v} \cdot \vec{u} ~ = ~
        v_{1}u_{1} + v_{2}u_{2} + v_{3}+u_{3} \]

        We learned that this scalar actually represented
        a lot more than it let on. In fact, the
        \textbf{scalar that results from a dot product}
        actually represents the product of both vectors'
        magnitudes as well as the cosine of the angle
        between them:
        \[ \vec{v} \cdot \vec{u} ~ = ~
        || \vec{v} || \cdot || \vec{u} || \cdot
        \cos{\theta} \]

        where, of course, $ \theta $ represents the angle
        between the vectors $ v $ and $ u $.
        \\
        \par We also see the angle between two vectors
        $ \theta $ represented by the following formula

        \[ \cos{\theta} =
        \frac{\vec{v} \cdot \vec{u}}{||\vec{v}|| ||\vec{u}||}
        \]

  \item We also learned about \textbf{projection},
        which is the idea of taking a vector $ v $ and
        then imposing it onto another vector $ u $.
        Imagine that we were just ``flattening'' a vector
        onto another, preserving its length/magnitude
        while maintaining another vector's direction.
        \[ proj_{v}u =
        \frac{\vec{v} \cdot \vec{u}}{\vec{v} \cdot \vec{v}} \vec{v} \]
        which can also be represented as
        \[ \Rightarrow \Biggr(
        \frac{\vec{u} \cdot \vec{v}}{||\vec{v}||}
        \Biggr)
        \cdot
        \Biggr(
        \frac{1}{||\vec{v}||}
        \Biggr) \vec{v}
        \]
        where the first term $ \frac{\vec{v} \cdot \vec{u}}{||\vec{v}||}$ represents the
        \textbf{scalar component of v} and the second
        term $ \Biggr( \frac{1}{||\vec{v}||}\Biggr) \vec{v} $
        represents the \textbf{unit vector of $ \vec{v} $}.


\end{itemize}

\section{Motivation}
In the former lectures, we have learned about
\textbf{vectors} as well as numerous operations
to perform ont hem

\begin{itemize}
  \item Vector Addition/Subtraction $ \rightarrow $ vectors
  \item Scalar Multiplication $ \rightarrow $ vector
  \item Dot Product $ \rightarrow $ scalar
\end{itemize}

What if there was such an operation that we were able
to \textbf{multiply} two vectors together?

\section{Cross Product}
\begin{center}
  \fbox{
    \parbox{\textwidth}{
      \begin{definition}
        Cross Product
      \end{definition}
      The cross product of two vecturs $ \vec{v} $
      and $ \vec{u} $, $ (\vec{v} \times \vec{u}) $, is
      the geometrically defined by the following vector:
      \[ \vec{v} \times \vec{u} := ||\vec{v}||||\vec{u}||
        \cdot \sin{\theta} \cdot \vec{n} \ \]
      where vector $ \vec{n} $ is the \textbf{normal unit
        vector} perpendicular to the plane spanned by
      $ \vec{v} $ and $ \vec{u} $, which is chosen
      accordingly by the right hand rule.

      \textbf{Layman Definition.}
      \\
      The cross product is the reuslting vector
      $ \vec{n} $, which is a vector that is orthogonal
      toe the plane of which $ \vec{v} $ and $ \vec{u} $
      occupy, and of which whose magnitude is determined
      by the product of the magnitude of $ \vec{v} $ and
      $ \vec{u} $, and the value of the angle betwen
      the two vectors that span the plane.
    }
  }
\end{center}
\textbf{TODO: Insert Picture from Tablet}
\subsection{Observations of the Cross Product}
TODO

\chapter{12.5: Lines and Planes in Space (04/07/23 - 04/10/23)}
\section{TODOs }
\begin{itemize}
  \item Finish writing lesson for lines in space
  \item Finish writing lesson for planes in space
  \item Finish OHW
\end{itemize}

\section{Lines in Space (04/07/23)}
TODO
\section{Reminders}
\begin{itemize}
\end{itemize}

\section{Objectives}
In this section, we want to be able to do the
following:
\begin{itemize}
  \item Be able to write equations for lines
        and line segments in $ \R^{3} $ space
        using scalar and vector products
\end{itemize}

\section{Motivation}
Recall in the last lessons we have been
learninge exclusively about vectors as well
as how we are able to manipulate them in order
to get different objects in space.
\begin{itemize}
  \item For example, we have learned that the
        \textbf{Dot Product} is a function that takes
        in two vectors $ \vec{v}$ and $ \vec{u}$
        and returns some scalar.
        \[ \vec{v} \cdot \vec{u}  \]
        Much like anything in multivariable calculus,
        we actually have to consider that a lot of the
        concepts we learn are \textbf{multi-dimensional}.
        In this case, we must understand that the
        \textbf{dot product} has both a \textbf{algebraic}
        and \textbf{geometric} definition.
        \par Algebraically, we can think of the dot product
        as the following equation:
        \[ \textrm{let $ \vec{v} = \langle v_{1}, v_{2}, v_{3} \rangle $}\]
        \[ \textrm{let $ \vec{u} = \langle u_{1}, u_{2}, u_{3} \rangle$}\]
        \[ \Rightarrow
        \vec{v} \cdot \vec{u} ~ = ~ v_{1}u_{1} + v_{2}u_{2}
        + v_{3}u_{3} \]
        \par Geometrically, we can think of the
        \textbf{dot product} as the \textbf{angle between
        two vectors, but scaled based on the magnitude
        of the vectors.}
        \[ \textrm{let $ \vec{v} = \langle v_{1}, v_{2}, v_{3} \rangle $}\]
        \[ \textrm{let $ \vec{u} = \langle u_{1}, u_{2}, u_{3} \rangle$}\]
        \[ \Rightarrow \vec{v} \cdot \vec{u} ~ = ~
        ||\vec{v}|| \cdot ||\vec{u}|| \cdot \cos{\theta}\]
        \par Of course, there are a few implications and
        use cases of the dot product both algebraically
        and geometrically
        \begin{itemize}
          \item If $ \vec{v} \cdot \vec{u} = 0 $,
                then we know that the vectors $ v $ and
                $ u $ are \textbf{orthogonal or
                perpendicular}, which makes sense,
                since if we were to find some value
                $ \cos{\theta} = 0$, then we would have
                $ \theta = \frac{\pi}{2} $, which is of
                course, a right angle.
                \item
        \end{itemize}

        \textbf{TODO: Include Graphics that Visualizes
        this relationship}

\end{itemize}

\subsection{2-D Lines versus 3-D Lines}
Whenever we were defining lines in $ \R^{2}$, we
were always thinking of these lines as an
a set of points with two defining characteristics:
\begin{itemize}
  \item Some point $ P_{0} $ that the line
        intersected
  \item Some slope or direction $ m $ that the
        line went in
\end{itemize}

With the intuition that lines in two-dimensional space
were defined by the points they intersectd as well
as their \textbf{slope}, which we can think of as
the \textbf{change in x and the change in y over time},
we can apply the same general principals to vectors in
three-dimensional space.
\subsection{3-D Lines}
Three dimensional lines, by comparison, are also
defined by some point that the line goes through
as well as a direction in which the line continues
infinitely. Instead of having a slope, however,
we like to think of the ``slope'' of a three-dimensional
line as a ``parallel'' vector, that doesn't necessarily
represent the actual line, but the \textbf{behavior}
of our current line.
\par A three-dimensional line is defined by the following
terms:
\begin{itemize}
  \item An \underline{\textbf{initial point}} $ P_{0} $ or
        just $ P $.
  \item A \underline{\textbf{vector}} that defines
        the line's \textbf{direction} and \textbf{behavior},
        $ \overrightarrow{P_{0}P} $ or $ \overrightarrow{PQ}$, where $ P_{0} $ and
        represents the initial point and $ P $ and $ Q $
        represent \textbf{any point on the line}.

\end{itemize}
\subsection{3-D Line Summary}
Essentially, much like hte \textbf{two-dimensional line},
a \textbf{three-dimensional line} is defined by some
point $ P $ that the line goes through, as well as
a \textbf{directional vector} that starts from that
initial point $ P $ and extends to any point $ Q $ on
the line $ \overrightarrow{PQ} $.



\section{Planes in Space (04/10/23)}
\section{Reminders}
TODo
\section{Objectives}
\begin{enumerate}
  \item Determine vector and component equations
  \item Produce non-zero vectors normal to a given plane
  \item Compute the distance from a point to a plane
        in space
  \item Determine whether two given planes coincide, intersect in a line, or are parallel

\end{enumerate}

\section{Motivation}
TODO
\chapter{11.6: Conic Sections (04/12/23)}
\section{Reminders}
\section{Objectives}
\section{Motivation}
\chapter{12.6: Cylinders and Quadric Surfaces (04/14/23)}
\section{Reminders}
\section{Objectives}
\begin{enumerate}
  \item Sketch the graph of various cylinders
  \item Graph the \textbf{six} quadric surfaces by hand
  \item Understand the usefulness of the coordinate plane
        traces as well as how to find them

\end{enumerate}

\section{Motivation}
\chapter{11.3: Polar Coordinates (04/17/23)}
\section{Reminders}
\section{Objectives}
\begin{enumerate}
  \item Be able to differentiate between poloar
        coordinates and Cartesian coordinates
  \item Be able to relate polar coordinates to
        Cartesian coordinates
  \item Graph polar coordinate functions

\end{enumerate}

\section{Motivation}
\chapter{13.1: Curves in Space and their Tangents (04/19/23)}
\section{Reminders}
\section{Objectives}
\section{Motivation}
\chapter{13.3: Arc Length}
\section{Reminders}
\section{Objectives}
\section{Motivation}
\chapter{14.1: Functions of Several Variables}
\section{Reminders}
\section{Objectives}
\section{Motivation}
\chapter{14.3: Partial Derivatives}
\section{Reminders}
\section{Objectives}
\section{Motivation}
\chapter{14.4: The Chain Rule}
\section{Reminders}
\section{Objectives}
\section{Motivation}
\chapter{14.5: Gradient Vectors and Tangent Planes}
\section{Reminders}
\section{Objectives}
\section{Motivation}
\chapter{14.5 (cont'd): Directional Derivatives}
\section{Reminders}
\section{Objectives}
\section{Motivation}
\chapter{14.6: Tangent Planes and Linearization}
\section{Reminders}
\section{Objectives}
\section{Motivation}
\chapter{10.9: Taylor's Formula}
\section{Reminders}
\section{Objectives}
\section{Motivation}
\chapter{10.9: Taylor's Polynomials}
\section{Reminders}
\section{Objectives}
\section{Motivation}
\chapter{14.7: Optimization}
\section{Reminders}
\section{Objectives}
\section{Motivation}
\chapter{14.8: Lagrange Multipliers}
\section{Reminders}
\section{Objectives}
\section{Motivation}
\chapter{14.8: Lagrange Multipliers (Part 2)}
\section{Reminders}
\section{Objectives}
\section{Motivation}
\end{sloppypar}
\end{document}
